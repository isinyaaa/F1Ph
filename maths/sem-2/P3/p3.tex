\documentclass{IMTexam}

\usepackage[enums]{IMTtikz}
\usepackage{booktabs}
\usepackage{stackengine,collcell}
\let\endminwd\relax
%\newcolumntype{L}[1]{>{\collectcell\xminwd l{#1}$}l<{$\endminwd\endcollectcell}}
\newcolumntype{C}[1]{>{\collectcell\xminwd c{#1}$}c<{$\endminwd\endcollectcell}}
%\newcolumntype{R}[1]{>{\collectcell\xminwd r{#1}$}r<{$\endminwd\endcollectcell}}
\def\minwd#1#2#3\endminwd{\stackengine{0pt}{#3}{\rule{#2}{0pt}}{O}{#1}{F}{F}{L}}
\newcommand\xminwd[1]{\minwd#1}

\givecredits
\author{Isabella B. Amaral}
\USPN{11810773}
\lecture{Matemática II}
\examname{Prova III}
\hwtype{Resolução}
\lcode{}
\date{10 de Agosto}

\newtheorem{definition}{Definição}
\newtheorem{theorem}{Teorema}[question]
\newtheorem{corollary}{Corolário}[theorem]
\newtheorem{lemma}[theorem]{Lema}
\let\oldemptyset\emptyset
\let\emptyset\varnothing
\newcommand\restrict[1]{{% we make the whole thing an ordinary symbol
        \left.\kern-\nulldelimiterspace % automatically resize the bar with \right
        % the function
        \vphantom{\big|} % pretend it's a little taller at normal size
        \right|_{#1} % this is the delimiter
}}
\DeclarePairedDelimiter\ceil{\lceil}{\rceil}
\DeclarePairedDelimiter\floor{\lfloor}{\rfloor}

\newcommand{\decSep}{,}
\newcommand{\decNum}[3]{\rlap{\ensuremath{\phantom{#1\mathord{\decSep}#2}\overline{\phantom{#3}}}}\num{#1.#2#3}}
%\newcommand{\decSI }[4]{\rlap{\ensuremath{\phantom{#1\mathord{\decSep}#2}\overline{\phantom{#3}}}}\SI{#1.#2#3}{#4}}


\begin{document}

    \maketitle

    \begin{questions}
        \question Suponha que $f\colon\mathbb{R}\longrightarrow\mathbb{R}$ seja
        contínua e prove que a EDO $y^{\prime\prime}-y=f(x)$ tem no máximo uma solução
        limitada.

        \begin{solution}
            \paragraph{Nota:} Essa solução foi inspirada por colegas.

            \begin{theorem}\label{thm:difSolution}
                Se $ \varphi_1\colon\mathbb{R}\longrightarrow\mathbb{R}$ e
                $\varphi_2\colon\mathbb{R}\longrightarrow\mathbb{R}$ compõe uma solução de
                $y^{\prime\prime}-y=f(x)$ então $\varphi_1-\varphi_2$ é solução de
                $y^{\prime\prime} - y = 0$.
            \end{theorem}

            \begin{proof}
                Pela linearidade do operador diferencial, segue que
                $(\varphi_1-\varphi_2)^{\prime\prime} =
                \varphi_1^{\prime\prime}-\varphi_2^{\prime\prime}$ de tal forma
                que, tomando a diferença
                \[ (\varphi_1-\varphi_2)^{\prime\prime}-(\varphi_1-\varphi_2)
                =\varphi_1^{\prime\prime}-\varphi_1-\del{\varphi_2^{\prime\prime}-\varphi_2}
                = f(x) - f(x)=0. \]
            \end{proof}

            \begin{lemma}\label{lem:difLimit}
                $\varphi(x) = a\,\mathrm{e}^x+b\,\mathrm{e}^{-x}$ é limitada
                se, e somente se, $a=b=0$.
            \end{lemma}

            \begin{proof}
                Para $a=b=0$ a expressão é limitada trivialmente (i.e. é
                constante e igual à zero). Para provar a volta, notamos
                primeiramente que $\varphi$ pode ser escrita como
                \[
                    \varphi(x)=\mathrm{e}^x\del{a+\dfrac{b}{\mathrm{e}^{2x}}} =
                    \mathrm{e}^{-x}\del{b+a\,\mathrm{e}^{2x}},
                \]
                de tal forma que, avaliando os limites da expressão
                \begin{align*}
                    \lim\limits_{x\to\infty} \varphi(x)&=\lim\limits_{x\to\infty}
                    \mathrm{e}^x\del{a+\dfrac{b}{\mathrm{e}^{2x}}}\\
                    \lim\limits_{x\to0^+} \varphi(x)&=\lim\limits_{x\to0^+}
                    \mathrm{e}^{-x}\del{b+a\,\mathrm{e}^{2x}},
                \end{align*}
                notamos que, para o primeiro limite, será $\pm\infty$ de acordo
                com o sinal de $a$, assim como no segundo caso, de acordo com o
                sinal de $b$, implicando que $\varphi$ não é limitada para
                $a\neq0$ ou $b\neq0$.
            \end{proof}

            Caso a equação admita duas soluções limitadas $f$ e $g$, por
            \ref{thm:difSolution} temos que $f-g$ também é solução limitada da
            equação homogênea associada.
            $y^{\prime\prime} - y=0$. Pelo teorema 8.6 do Apostol, toda solução
            da equação homogênea é da forma
            $a\,\mathrm{e}^x+b\,\mathrm{e}^{-x}$, com $a,b\in\mathrm{R}$,
            então, por \ref{lem:difLimit}, a única solução limitada dessa
            equação é a função identicamente nula. Dessa forma, temos
            $f-g=0\implies f=g$, i.e. deve existir somente uma solução
            limitada.
        \end{solution}

        \question Determine a família de curvas ortogonais às circunferências
        do plano que passam pelos pontos $(4,0)$ e $(-4,0)$.

        \begin{solution}
            Sendo a circunferência simétrica, temos que seu centro deve estar
            localizado entre os pontos $(4,0)$ e $(-4,0)$ e, portanto, no eixo
            $y$. Seja, então, $P=(0,C)$ o centro da circunferência, devemos ter
            um raio que obedeça $R^2=4^2+C^2$ e, também a equação da
            circunferência a seguir
            \begin{equation}\label{eq:circEq}
            x^2+(y-C)^2=16+C^2.
            \end{equation}
            A partir desta podemos encontrar as retas tangentes à
            circunferência, utilizando sua derivada
            \begin{equation}\label{eq:circDif}
                \dod{\sbr{x^2+(y-C)^2}}{x}=\dod{\sbr{16+C^2}}{x}\implies
                2x+2(y-C)y'=0,
            \end{equation}
            e, isolando $C$ em \ref{eq:circEq}, temos
            \begin{equation}\label{eq:circCenter}
                x^2+(y-C)^2=16+C^2\implies C=\dfrac{x^2+y^2-16}{2y}.
            \end{equation}
            O que nos permite resolver \ref{eq:circDif} em termos mais gerais:
            \begin{align*}
                x+y\,y'-\del{\dfrac{x^2+y^2-16}{2y}}y'&=0\\
                \implies\del{x^2-y^2-16}y'-2x\,y&=0.
            \end{align*}
            De tal forma que as trajetórias ortogonais devem obedecer
            \begin{align*}
                y'&=\dfrac{2x\,y}{y^2-x^2+16}
                \intertext{e, por frações parciais, temos}
                &=\dfrac{x^2-16}{2x\,y}-\dfrac{y}{2x}\\
                \implies y'+\dfrac{1}{2x}y&=\dfrac{x^2-16}{2x}y^{-1}.
            \end{align*}

            É sabido que uma EDO como a acima, da forma
            $y'+P(x)y=Q(x)y^n+q\,n\neq0$ e $n\neq1$ (i.e. de Bernoulli) pode
            ser transformada em uma equação linear de primeira ordem por uma
            substituição em $y$: $v=y^k$, com $k=1-n$, como foi
            visto na questão 13 da seção 8.5 do Apostol (veja \ref{thm:apostol8.6.13}).

            Sendo $n=-1$, temos $k=2$, o que implica em
            \[
                v'+2\del{\dfrac{1}{2x}}v=2\del{\dfrac{x^2-16}{2x}}\implies
                v'+\dfrac{1}{x}v=x-\dfrac{16}{x},
            \]
            e, pelo teorema 8.3 do Apostol, temos que a equação $y'+P(x)y=Q(x)$
            com $f(a)=b$ no intervalo aberto $I$ tem como solução
            \[
                f(x)=b\,\mathrm{e}^{-A(x)}+\mathrm{e}^{-A(x)}
                \int_a^xQ(t)\mathrm{e}^{A(t)}\dif t\quad\text{onde }A(x)=\int_a^x P(t)\dif t.
            \]
            Substituindo por valores, temos que
            $v'+\dfrac{1}{x}v=x-\dfrac{16}{x}$, $ P(x)=\dfrac{1}{x}$ e
            $Q(x)=x-\dfrac{16}{x}$. Assim:
            \[ A(x)=\int_a^x \dfrac{1}{t}\dif t=\log\del{\dfrac{x}{a}}\implies f(x)=b\del{\dfrac{x}{a}}+\del{\dfrac{x}{a}}\int_a^x t-\dfrac{16}{t}\del{\dfrac{x}{a}}\dif t \]
            Portanto, temos
            \[ y^2=v=\dfrac{b\,a}{x}+16x\dfrac{\log (x/a)}{a}+\dfrac{a^2-x^2}{2}. \]
        \end{solution}

        \question Considere a EDO
        \[ y'=\dfrac{2y^2+x}{3y^2+5}, x\in\mathbb{R}, \]
        e seja $\varphi(x), x\in\mathbb{R}$ a única solução desta equação tal que
        $\varphi(0)=0$.

        \begin{parts}
            \part Decidir se 0 é ponto de máximo, de mínimo ou não é um ponto
            de extremo local de $\varphi$.

            \begin{solution}
                Plugando a condição inicial na equação, temos $\varphi'(0)=0$,
                o que implica em $0$ ser um ponto crítico de $\varphi$. Além
                disso, por ser composta de duas funções deriváveis, $\varphi'$
                é derivável e, pela regra do quociente, temos
                \[
                    \varphi^{\prime\prime} =
                    \dfrac{
                        \del{4\varphi\,\varphi'+1}\del{3\varphi^2+5} -
                        \del{2\varphi^2+x}6\varphi\,\varphi'}
                    {\del{3\varphi^2+5}^2}.
                \]
                Dessa forma, temos $\varphi^{\prime\prime}(0)=1/5>0$ e,
                portanto, $0$ é ponto de mínimo local de $\varphi$.
            \end{solution}

            \part Determine reais $a>0$ e $b>0$ tais que $\varphi(x)> ax -b$, para
            todo $x \geqslant \sfrac{10}{3}$. (Mostre inicialmente que
            $\varphi'(x)\geqslant \sfrac{2}{3}$, se $x\geqslant \sfrac{10}{3}$.)

            \begin{solution}
                Observe que, se $x\geqslant 10/3$, então
                \[ 2\varphi^2(x)+x\geqslant
                2\varphi^2(x)+\dfrac{10}{3}=\dfrac{2}{3}\del{3\varphi^2(x)+5},
                \]
                e, portanto, $\varphi'(x)>0$. Nessa condição,
                $\varphi(x)$ será estritamente maior que a reta cujo
                coeficiente angular é $a=2/3$ e que possui $x=10/3$ como raiz, já
                que $\varphi(10/3)>0$ inequivocadamente, o que nos dá uma reta
                da forma $g(x)=2x/3-20/9$ (i.e. $b=20/9$), e temos a desigualdade
                $\varphi(x\geqslant 10/3)>2x/3-20/9$.

                \hfill\qedsymbol
            \end{solution}

            \part Prove que $\lim\limits_{x\to\infty}\dfrac{x}{\varphi^2(x)}=0$.

            \begin{solution}
                Tomamos, novamente, $a=2/3$ e $b=20/9$. Do item anterior, temos que
                $\varphi(x)>a\,x-b\geqslant0$ para todo $x\geqslant 10/3$.
                Segue, então, que
                $\varphi^2(x\geqslant 10/3)>\del{a\,x-b}^2=a^2x^2-2a\,b\,x+b^2$, e, dessa forma, temos
                \[ 0<\dfrac{x}{\varphi^2(x)}<\dfrac{x}{a^2x^2-2a\,b\,x+b^2}, \]
                e, no limite
                \[ 0\leqslant \lim\limits_{x\to\infty}
                \dfrac{x}{\varphi^2(x)}\leqslant \lim\limits_{x\to\infty}
                \dfrac{1}{a^2x-2a\,b+b^2/x}=0, \]
                e, pelo teorema do confronto, $\lim\limits_{x\to\infty}\dfrac{x}{\varphi^2(x)}=0$.
            \end{solution}

            \part Calcular $\lim\limits_{x\to\infty}\dfrac{\varphi(x)}{x}$.

            \begin{solution}
                Considere $x\geqslant 10/3$ e, portanto, $\varphi(x)>0$.
                Tomemos $\varphi'(x)$ na forma
                \[ \varphi'(x)=\dfrac{2+x/\varphi^2(x)}{3+5/\varphi^2(x)}, \]
                segue daí que
                \[ \lim\limits_{x\to\infty}
                \varphi(x)=\dfrac{2+\lim\limits_{x\to\infty}x/\varphi^2(x)}
                {3+\lim\limits_{x\to\infty}5/\varphi^2(x)}.
                \]

                Sendo
                \[ \lim\limits_{x\to\infty}\varphi(x)=\lim\limits_{x\to\infty}\varphi^2(x)=+\infty\]
                já que $\varphi(x\geqslant 10/3)>a\,x-b$ e, então, como
                $\lim\limits_{x\to\infty}a\,x-b=+\infty$ segue que
                $\lim\limits_{x\to\infty} \varphi(x)=+\infty$.

                Notamos, então, que
                \[ \lim\limits_{x\to\infty} \varphi(x)=+\infty \implies
                \lim\limits_{x\to\infty} \varphi^2(x)=+\infty. \]

                Temos, então, que $\lim\limits_{x\to\infty}\varphi'(x)=2/3$.
                Ora, como $\lim\limits_{x\to\infty}$ existe e é igual a $2/3$
                e
                $\lim\limits_{x\to\infty}\varphi(x)=\lim\limits_{x\to\infty}x=+\infty$,
                segue-se da regra de l'hôpital que
                $\lim\limits_{x\to\infty}\varphi(x)/x=2/3$.
            \end{solution}

        \end{parts}

        \question Considere $p\colon\mathbb{R}\longrightarrow\mathbb{R}$ e
        $q\colon\mathbb{R}\longrightarrow\mathbb{R}$ funções contínuas e
        considere $u$ e $v$ soluções de $y^{\prime\prime}+p(x)y'+q(x)y=0$ com
        $u(0)=1,u'(0)=0,v(0)=0,v'(0)=1$.

        \begin{parts}
            \part Prove uqe, se $\alpha$ e $\beta$ são reais, a solução de
            $y^{\prime\prime}+p(x)y'+q(x)y=0,y(0)=\alpha,y'(0)=\beta$ é
            $y(x)=\alpha\,u(x)+\beta\,v(x)$.

            \part Seja $a\in\mathbb{R}$. Prove que se $u(a)=0$ então
            $u'(a)\neq0$, prove também que $v(a)\neq0$.

            \part Se $x_1<x_2$ são tais que $u(x_1)=u(x_2)=0$ e $u(x)\neq0$,
            para todo $x\in\intoo{x_1,x_2}$, então existe um, e só um,
            $\bar{x}\in\intoo{x_1,x_2}$ tal que $v(\bar{x})=0$.

        \end{parts}

        \question Considere $0<a_1\leqslant b_1$ e as sequências $(a_n)$ e
        $(b_n)$ definidas como $a_{n+1}=\sqrt{a_n\,b_n}$ e
        $b_{n+1}=\dfrac{a_n+b_n}{2}$.

        \begin{parts}
            \part Mostre que essas sequências convergem.

            \begin{solution}
                Provemos por indução que, para todo $n\in\mathbb{N}$, se tem
                \[ a_1\leqslant a_n\leqslant a_{n+1}\leqslant b_{n+1}\leqslant b_n\leqslant b_1. \]
                A partir do caso base (i.e. $a_1\leqslant b_1$), temos, pela
                desigualdade das médias, que $ a_1\leqslant\sqrt{a_1\,b_1}
                \leqslant (a_1+b_1)/2\leqslant b_1$ e segue que $a_1\leqslant
                a_2\leqslant b_2\leqslant b_1$.

                Supondo que a equação acima seja válida para algum $n\geqslant
                1$ temos, pela desigualdade das médias,
                \[
                    a_{n+1} \leqslant \sqrt{a_{n+1}b_{n+1}}\leqslant
                    (a_{n+1}+b_{n+1})/2\leqslant b_{n+1}
                \]
                uma vez que $a_{n+1}\leqslant b_{n+1}$, de onde
                concluímos que $a_1\leqslant a_{n+1}\leqslant a_{n+2}\leqslant
                b_{n+2}\leqslant b_{n+1} \leqslant b_{1}$. A partir desse fato,
                segue que ambas as sequências $(a_n)$ e $(b_n)$ são
                monotônicas e limitadas e, portanto, convergem.
            \end{solution}

            \part Mostre que o limite dessas sequências é o mesmo.

            \begin{solution}
                Pela definição do termo geral da sequência $(b_n)$, temos
                \[ b_{n+1}=\dfrac{a_{n}+b_n}{2} \]
                ou seja,
                \[ 2b_{n+1}=a_{n}+b_n. \]
                Como toda subsequência de uma sequência converge para o mesmo
                limite, tomando $n\to\infty$ dos dois lados da igualdade
                segue que $2B-B=A$, isto é, $B=A$.
            \end{solution}

        \end{parts}

        \question Sejam $p>0, q>0$ reais e considere, para os naturais
        $n\geqslant
        1,x_n=\dfrac{(p+1)(p+2)\cdots(p+n)}{(q+1)(q+2)\cdots(q+n)}$. Prove que
        $\sum_{n=1}^\infty x_n$ converge se, e só se, $q>p+1$.

        \begin{solution}
            \paragraph{Nota:} Essa solução foi inspirada por colegas.

            \begin{theorem}\label{thm:divConv}
                Seja $\sum x_n$ uma série de termos positivos. Se existem $r>0$ e $n_0\in\mathbb{N}$ tais que
                \[ \dfrac{x_{n+1}}{x_n}\leqslant 1-\dfrac{1+r}{n} \]
                para todo $n>n_0$, então a série $\sum x_n$ converge. Por outro lado, se
                \[ \dfrac{x_{n+1}}{x_n}\geqslant 1-\dfrac{1}{n} \]
                para todo $n>n_0$, então a série $\sum x_n$ diverge.
            \end{theorem}
            \begin{proof}
                Observe que $x_{n+1}/x_n\leqslant 1-(1+r)/n$ equivale a
                \[ n\dfrac{x_{n+1}}{x_n}\leqslant n-1-r, \]
                isto é,
                \[ x_n\leqslant\dfrac{(n-1)x_n-n\,x_{n+1}}{r}. \]
                As reduzidas da série cujo termo geral é
                $\del{(n-1)x_n-n\,x_{n+1}}/r$ formam uma série telescópica, da qual segue
                \[ \sum_{k=n_0+1}^n x_k\leqslant
                \dfrac{n_0x_{n_0}}{r}-\dfrac{n\,x_{n+1}}{r}\leqslant
                \dfrac{n_0x_{n_0}}{r} \]
                para todo $n>n_0$. Dessa forma, $\sum x_n$ é uma série de
                termos positivos cujas reduzidas são limitadas e, portanto,
                converge. Para provar o segundo resultado, note que
                $x_{n+1}/x_n\geqslant 1-1/n$ nos diz que
                \[ n\,x_{n+1}\geqslant (n-1)x_n \]
                se $n > n_0$, donde concluímos que $(n-1)x_n\geqslant
                n_0x_{n_0}+1$ para todo $n>n_0$. Logo, temos que
                \[ \sum_{k=n_0+1}^n x_k\geqslant
                n_0x_{n_0}+1\sum_{k=n_0+1}^n\dfrac{1}{k-1}=n_0x_{n_0+1}\sum_{k=n_0}^{n-1}\dfrac{1}{n}
                \]
                para todo $n>n_0$. Como a série harmônica é divergente, segue
                que $\sum x_n$ é uma série divergente nesse caso.
            \end{proof}

            Sendo os termos de $x_n$ todos positivos, e
            \[
                \dfrac{x_{n+1}}{x_n}=\dfrac{p+n+1}{q+n+1} =
                1+\dfrac{p-q}{q+n+1}=1-\dfrac{q-p}{q+n+1},
            \]
            ou seja,
            \[ \dfrac{x_{n+1}}{x_n}\leqslant 1-\dfrac{(q-p-1)+1}{n} \]
            para todo $n>\floor{q+1}$, se $p>q+1$, então pelo
            teorema \ref{thm:divConv} a série $\sum x_n$ converge. Reciprocamente, se $\sum
            x_n$ converge, afirmamos que $q>p+1$. Com efeito, se
            $q<p+1$, teríamos $x_n>\dfrac{p+n}{q+1}$ para todo
            $n\in\mathbb{N}$ e, então a série seria claramente divergente.
            No caso em que $q=p+1$, seria
            \[ \dfrac{x_{n+1}}{x_n}\geqslant 1-\dfrac{1}{n} \]
            para todo $n>\ceil{q+1}$, e portanto, do teorema \ref{thm:divConv}, teríamos
            $\sum x_n$ divergente, como se queria demonstrar.
        \end{solution}

        \question Decidir se as séries abaixo convergem e se convergem
        absolutamente:

        \begin{parts}
            \part \[\sum_{n=1}^\infty
            \dfrac{(-1)^{n-1}}{\log\del{\mathrm{e}^n-\mathrm{e}^{-n}}}.\]

            \begin{solution}
                Sendo $1>\mathrm{e}^{-n}$ para todo $n\in\mathbb{N}$, segue que
                $\mathrm{e}^n-1<\mathrm{e}^n-\mathrm{e}^{-n}$. Assim, visto que
                $\lim(\mathrm{e}^n-1)=+\infty$, obtemos
                $\lim(\mathrm{e}^n-\mathrm{e}^{-n})=+\infty$, o que implica
                $\lim\log(\mathrm{e}^n-\mathrm{e}^{-n})=+\infty$. Dessa forma,
                $\lim\dfrac{1}{\log(\mathrm{e}^n-\mathrm{e}^{-n})}=0$ e,
                portanto, pelo teste de Leibniz, $\sum
                \dfrac{(-1)^{n-1}}{\log(\mathrm{e}^n-\mathrm{e}^{-n})}$
                converge. Por outro lado,
                $\mathrm{e}^n>\mathrm{e}^n-\mathrm{e}^{-n}$, de onde
                $n>\log(\mathrm{e}^n-\mathrm{e}^{-n})$ para
                todo $n$ natural, ou seja
                $1/n<1/\log(\mathrm{e}^n-\mathrm{e}^{-n})$. Como a
                série harmônica é divergente, pelo
                teste da comparação temos que
                $\sum 1/\log\del{\mathrm{e}^n-\mathrm{e}^{-n}}$
                diverge. Concluímos, então, que a série
                $\sum\dfrac{(-1)^{n-1}}{\log\del{\mathrm{e}^n-\mathrm{e}^{-n}}}$
                é condicionalmente convergente.
            \end{solution}

            \part \[\sum_{n=2}\sin\del{n\,\pi+\dfrac{1}{n\,\log n}}.\]

            \begin{solution}
                Subdividindo em casos, temos
                \begin{enumerate}
                    \item Para $n=2m$, temos
                    \[ \sin\del{2m\pi+\dfrac{1}{n\log n}} = \sin\del{\dfrac{1}{n\log n}}, \]
                    \item e, para $n=2m-1$,
                    \begin{align*}
                        \sin\del{(2m-1)\pi + \dfrac{1}{n\log
                        n}}&=\sin\del{\dfrac{1}{n\log n} -
                        \pi}=-\sin{\pi-\dfrac{1}{n\log n}}\\
                        &=-\cos\del{\dfrac{\pi}{2}-\dfrac{1}{n\log
                        n}}=\sin\del{-\dfrac{1}{n\log n}}\\
                        &=-\sin\del{\dfrac{1}{n\log n}}.
                    \end{align*}
                \end{enumerate}
                Portanto, a série $\sum_{n=2}^\infty \sin\del{n\pi
                +\dfrac{1}{n\log n}}$ é equivalente à série
                \[ \sum_{n=2}^\infty (-1)^n\sin\del{\dfrac{1}{n\log n}}. \]
                Como $\lim\limits_{n\to \infty}\sin\del{\dfrac{1}{n\log n}}=0$,
                pelo teste de Leibniz temos que a série converge. Por outro
                lado, como $\sum 1/(n\log n)$ diverge e
                \[ \lim\limits_{x\to\infty} \dfrac{\sin\del{1/(n \log n)}}{1/(n
                \log n)}=\lim\limits_{u\to0^+}\dfrac{\sin u}{u}=1, \]
                temos que a série $\sum_{n=2}^\infty \sin\del{n\pi + 1/(n \log
                n)}$ é condicionalmente convergente.
            \end{solution}

        \end{parts}

        \question Determine o(s) $C\in\mathbb{R}$ para o(s) qual(is) a função
        $f(x)=\sfrac{C\,x}{x^2+1}-\sfrac{1}{2x-1}$ é integrável em
        $\intco{1,+\infty}$ e, nesse(s) caso(s), calcule sua integral nesse
        intervalo.

        \begin{solution}
            Queremos saber se a integral
            \[ \int_1^\infty\del{\dfrac{C\,x}{x^2+1}-\dfrac{1}{2x-1}}\dif
            x=\int_1^\infty
            \dfrac{\del{2C-1}x^2+C\,x-1}{\del{2x-1}\del{x^2+1}}\dif x \]
            existe. Para isso, podemos usar o critério de comparação limite
            (teorema 10.25 do Apostol):
            \begin{align*}
                \lim\limits_{x\to\infty}
                \dfrac{\dfrac{\del{2C-1}x^2+C\,x-1}{\del{2x-1}
                \del{x^2+1}}}{1/x}&=\lim\limits_{x\to\infty}
                \del{\dfrac{x}{2x-1}}
                \lim\limits_{x\to\infty}\del{\dfrac{x^2(2C-1)+C\,x-1}{x^2+1}}\\
                &=\dfrac{1}{2}\lim\limits_{x\to\infty}\del{\dfrac{x^2(2C-1)+C\,x-1}{x^2+1}}\\
                &=\dfrac{1}{2}\lim\limits_{x\to\infty}\del{\dfrac{2C-1+C/x-1/x^2}{1+1/x^2}}\\
                &=\dfrac{2C-1}{2}.
            \end{align*}

            Portanto, como $\int_1^\infty 1/x \dif x$ diverge, a integral
            proposta certamente irá divergir para todo $C$ tal que
            $(2C-1)/2\neq0$. Então, o caso de convergência deve ser para
            $C=1/2$. Substituindo, temos:
            \begin{align*}
                \int_1^\infty \del{\dfrac{x/2}{x^2+1}-\dfrac{1}{2x-1}}\dif
                x&=\lim\limits_{a\to\infty}\del{\dfrac{1}{2}\int_1^a\dfrac{x}{x^2+1}\dif
                x-\int_1^a \dfrac{1}{2x-1}\dif x}\\
                &= \lim\limits_{a\to\infty} \eval{\del{\dfrac{\log(x^2+1)}{4}-\dfrac{\log|2x-1|}{2}}}_1^a\\
                &= \dfrac{1}{4}\lim\limits_{x\to\infty}\sbr{\del{\log\del{a^2+1}-2\log(2a-1)}-\del{\log 2-2\log 1}}\\                &= \dfrac{1}{4}\lim\limits_{x\to\infty}\sbr{\log\del{\dfrac{a^2+1}{2(2a-1)^2}}}\\
                &= \dfrac{1}{4}\log\sbr{\lim\limits_{x\to\infty}\del{\dfrac{1+1/a^2}{8-8/a+2/a^2}}}\\
                &= \dfrac{1}{4}\cdot\log\del{\dfrac{1}{8}}=\dfrac{-3\log2}{4}.
            \end{align*}
        \end{solution}

        \question Determine os $x\in\mathbb{R}$ para os quais as séries abaixo convergem:

        \begin{parts}
            \part \[ \sum_{n=1}^\infty \dfrac{(-1)^{n+1}\,2^n\,\sin^{2n}x}{n}. \]

            \begin{solution}
                Como $\sum (-1)^{n+1}/n$ converge, pelo teste de Abel, a série
                \[ \sum \del{(-1)^{n+1}2^n \sin^{2n}x}/n \]
                converge se $x_n=2^n\sin^{2n}x$ é uma sequência não crescente.
                Mas $x_n=(2\sin^2 x)^n$ é não crescente se, e somente se,
                $2\sin^2 x\leqslant 1$, isto é, se $|\sin x| \leqslant \sqrt2/2$.
                Portanto, se
                \[
                    -\pi/4+2k\,\pi\leqslant x \leqslant \pi/4+2k\pi\quad(k\in\mathbb{Z}),
                \]
                então a série $\sum (-1)^{n+1}/n$ é
                convergente.
            \end{solution}

            \part \[ \sum_{n=1}^\infty\dfrac{2^n\,\sin^n x}{n^2}. \]

            \begin{solution}
                Como $\sum 1/n^2$ converge, pelo teste de Abel, a série
                \[
                    \sum \del{2^n \sin^n x}/n^2
                \]
                converge se $x_n=2^n\sin^n x$ é não crescente. Mas $x_n=(2\sin
                x)^n$ é não crescente se, e somente se, $0\leqslant \sin
                x\leqslant 1/2$. Portanto, se
                \[
                    2k\,\pi\leqslant x \leqslant \pi/6 + 2k\,\pi\quad (k\in\mathbb{Z}),
                \]
                a série $\sum \del{2^n\sin^n x}/n^2$ é convergente.
            \end{solution}

        \end{parts}

        \question Seja, para $n\in\mathbb{N}, n\geqslant 1$ e
        $\alpha\in\intoo{0,+\infty}$ $f_n(x)=\dfrac{\cos{n\,x}}{n^\alpha},x\in
        \mathbb{R}$.

        \begin{parts}
            \part Prove que, se $\alpha>1$ então $\sum_{n=1}^\infty f_n(x)$
            converge uniformemente em $\mathbb{R}$.

            \begin{solution}
                \paragraph{Nota: } feita nos 45 do segundo tempo, espero que dê pra aproveitar algo :).

                Sabemos que
                \[ 0\leqslant \dfrac{|\cos(n\,x)|}{n^\alpha} \leqslant\dfrac{1}{n^\alpha} \]
                para todo $n\in\mathbb{N}$, de tal forma que o somatório $\sum
                f_n(x)$ converge uniformemente em $\mathbb{R}$, já que $\sum
                \dfrac{1}{n^\alpha}$ converge para $\alpha>1$ como fora
                demonstrado em aula.
            \end{solution}

            \part Mostre que se $0<\alpha\leqslant 1$ então $\sum_{n=1}^\infty
            f_n(x)$ não converge em $\mathbb{R}$.

            \begin{solution}
                Tomando $x=2k\pi, k\in\mathbb{Z}$, temos
                \[\sum_{n=1}^\infty \dfrac{\cos(nx)}{n^\alpha}=\sum_{n=1}^{\infty} \dfrac{1}{n^\alpha}, \]
                e, portanto, se $x$ é múltiplo de $2\pi$, a série
                $\sum_{n=1}^\infty \cos(nx)/n^\alpha$ não converge, pois
                $\sum_{n=1}^{\infty} 1/n^\alpha$ diverge para
                $\alpha\in\intoc{0,1}$ como mostrado em aula, o que implica que
                $\sum f_n(x)$ não converge para estes valores de $\alpha$.
            \end{solution}

            \part Tome $\alpha>1$ e considere
            $F_\alpha\colon\mathbb{R}\longrightarrow\mathbb{R}$ o limite de
            $\sum_{n=1}^\infty f_n(x)$. Calcule $\int_0^\pi F_\alpha(x)\dif x$
            e $\int_0^{\pi/2}F_\alpha (x)\dif x$.

            \begin{solution}
                Pelo teorema 11.4 do Apostol, temos que
                \[ \int_0^x F_\alpha(u)\dif u=\sum_{n=1}^\infty\int_0^x
                f_n(u)\dif
                u=\sum_{n=1}^\infty\int_0^x\dfrac{\cos(nu)}{n^\alpha}=\sum_{n=1}^\infty\dfrac{\sin(nu)}{n^{\alpha+1}}.
                \]
                Para $x=\pi$, temos
                \[ \sum_{n=1}^\infty \dfrac{\sin(nu)}{n^{\alpha+1}}=0. \]
                Para $x=\pi/2$, temos
                \[ \sum_{n=1}^\infty \dfrac{\sin(nu)}{n^{\alpha+1}} \]
                convergente, já que se $\alpha>1$ então
                \[ \sum_{n=1}^\infty\int_0^x f_n(u)\dif u \]
                converge uniformemente em $\mathbb{R}$.

                Sabemos que
                \[ \int_0^x \dfrac{\cos(nu)}{n^\alpha}\dif u=\dfrac{\sin(nu)}{n^{\alpha+1}}, \]
                portanto
                \[ \sum_{n=1}^{\infty}\int_0^x \dfrac{\cos(nu)}{n^\alpha}\dif u=\sum_{n=1}^{\infty}\dfrac{\sin(nu)}{n^{\alpha+1}}, \]
                Para todo $n\in\mathbb{N}$ e $x\in\mathbb{R}$ vale, então, que
                \[0\leqslant \dfrac{|\sin(nx)|}{n^{\alpha+1}}\leqslant \dfrac{1}{n^{\alpha+1}} \]
                A série converge uniformemente em $\mathbb{R}$ já que $\sum
                1/n^{\alpha+1}$ é convergente para $\alpha>1$, como fora
                demonstrado em aula.
            \end{solution}

            \part Considere $\alpha>2$ e prove que $F_\alpha$ é derivável.
            Calcule, nesse caso, $F'_\alpha(x)$.

            \begin{solution}
                $F_\alpha$ é derivável, já que para $f_n$ de classe $C^1$ para
                todo $n\in\mathbb{N}$ onde $\sum f'_n(x)$ converge
                uniformemente em $\mathbb{R}$, $F_\alpha$ é derivável, e
                $F'_\alpha(x)=\sum_{n=1}^{\infty}f'_n(x)$. Demonstração:
                Pelo teorema fundamental do cálculo, para todo $n\in\mathbb{N}$ e $x\in\mathbb{R}$, vale que
                \[ \sum_{k=1}^{n}f_k(x)=\sum_{k=1}^{n}f_k(0)+\int_0^x\sum_{k=1}^{n}f'_k(t)\dif t. \]
                Tomando o limite, pelo teorema 11.3 do Apostol, temos que
                \[ F_\alpha(x)=F_\alpha(0)+\int_0^x\sum_{n=1}^{\infty}f'_n(t)\dif t \]
                já que $\sum f'_n$ converge uniformemente em $\mathbb{R}$.
                Pelo teorema 11.1 do Apostol, $\sum_{n=1}^{\infty} f'_n$ é contínua, logo $F_\alpha$ é derivável e $F'_\alpha(x)=\sum_{n=1}^{\infty}f'_n(x)$.
                Portanto, $F'_\alpha(x)=\sum_{n=1}^{\infty}\dfrac{\sin(nx)}{n^{\alpha-1}}$.
            \end{solution}

        \end{parts}

        \extra{Apêndice}

        \begin{theorem}\label{thm:apostol8.6.13}
            Seja $k$ uma constante não nula. Assuma que $P$ e $Q$ são
            contínuos num intervalo $I$. Se $a\in I$ e $b$ é qualquer
            número real, seja $v=g(x)$ a única solução do problema de valor
            inicial dado por
            \begin{equation}\label{eq:thmEq}
                v'+k\,P(x)v=k\,Q(x)
            \end{equation}
            em $I$, com $g(a)=b$. Se $n\neq1$ e $k=1-n$, prove que uma
            função $y=f(x)$, que nunca é nula em $I$, é solução pra o
            problema de valor inicial dado por
            \[ y'+P(x)y=Q(x)y^n \quad\textup{em }I,\textup{ com }f(a)^k=b \]
            se, e somente se, a $k-$ésima potência de $f$ é igual à $g$ em $I$.
        \end{theorem}
        \begin{proof}
            Tomemos
            \[ v=y^k\implies v'=y'k\,y^{k-1}, \]
            substituindo na equação \ref{eq:thmEq}, temos
            \begin{align*}
                y'k\,y^{k-1}+k\,P\,y^k&=k\,Q
                \intertext{e, como $k=1-n$}
                y'\,y^{-n}+P\,y^{1-n}&=Q\\
                \implies y'+P\,y&=Q\,y^n
            \end{align*}
        \end{proof}
    \end{questions}
\end{document}
