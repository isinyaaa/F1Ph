\documentclass[
    % -- opções da classe memoir --
    12pt,
    openright,
    twoside,
    a4paper,
    % -- opções do pacote babel --
    english,
    french,
    spanish,
    brazil
    ]{abntex2}

% Pacotes básicos
\usepackage{lmodern}
\usepackage[T1]{fontenc}
\usepackage[utf8]{inputenc}
\usepackage{lastpage}
\usepackage{indentfirst}
\usepackage{color}
\usepackage{graphicx}
\usepackage{microtype}

% Pacotes de citações
\usepackage[brazilian,hyperpageref]{backref}
\usepackage[alf]{abntex2cite}

% Configurações do pacote backref
% Usado sem a opção hyperpageref de backref
\renewcommand{\backrefpagesname}{Citado na(s) página(s):~}
% Texto padrão antes do número das páginas
\renewcommand{\backref}{}
% Define os textos da citação
\renewcommand*{\backrefalt}[4]{
    \ifcase #1 %
        Nenhuma citação no texto.%
    \or
        Citado na página #2.%
    \else
        Citado #1 vezes nas páginas #2.%
    \fi}

% Informações de dados para CAPA e FOLHA DE ROSTO
\titulo{Ensaio: sobre a evolução convergente do cérebro cetáceo}
\autor{Isabella Basso do Amaral}
\local{Brasil}
\data{18 de Janeiro de 2022}
\renewcommand{\orientadorname}{Professor:}
\orientador{Diogo Meyer}
%\coorientador{Equipe \abnTeX}
\instituicao{%
    Universidade de São Paulo -- USP
    \par
    INOVA
    \par
    Graduação em Ciências Moleculares}
\tipotrabalho{Ensaio}

% O preambulo deve conter o tipo do trabalho, o objetivo,
% o nome da instituição e a área de concentração
\preambulo{}

% alterando o aspecto da cor azul
\definecolor{blue}{RGB}{41,5,195}

\usepackage{csquotes}

% informações do PDF
\makeatletter
\hypersetup{
    %pagebackref=true,
    pdftitle={\@title},
    pdfauthor={\@author},
    pdfsubject={\imprimirpreambulo},
    pdfcreator={Isabella Basso do Amaral},
    colorlinks=true,
    linkcolor=blue,
    citecolor=blue,
    filecolor=magenta,
    urlcolor=blue,
    bookmarksdepth=4
}
\makeatother

% O tamanho do parágrafo é dado por:
\setlength{\parindent}{1.3cm}

% Controle do espaçamento entre um parágrafo e outro:
\setlength{\parskip}{0.2cm}  % tente também \onelineskip

\makeindex

\begin{document}

\frenchspacing

% \pretextual

%\imprimircapa

\imprimirfolhaderosto*

\textual

\textbf{%
    Geralmente adotava-se que os cetáceos, embora cognitivamente complexos, não
    possuíam um cérebro bem diferenciado, sendo tido como homogêneo%
    \footnote{Veja \cite{glezer1988implications}}. Uma análise mostra que esse
    não é o caso.
}

Os cetáceos têm uma longa e (dramaticamente) divergente história evolutiva em
relação aos mamíferos terrestres, o que é intuitivo haja vista de que são
animais aquáticos.

Ao longo de seus $\approx 50$ milhões de anos de evolução
\cite{gingerich1998likelihood}, eles adquiriram um conjunto altamente
especializado de características neurobiológicas, que incluem capacidades
distintas daquelas dos primatas, como a ecolocalização, presente na subordem
dos odontocetos \cite{SensoryS50:online}.

Eles, no entanto, possuem um conjunto de atributos cognitivos que são
surpreendentemente convergentes com os de muitos primatas, incluindo grandes
símios e humanos, como por exemplo capacidades auditivas e comunicativas
complexas, além de organização social complexa \cite{marino2002convergence,
sayigh2014cetacean}.

Dessa forma, os cérebros dos cetáceos oferecem uma oportunidade crítica para
abordar questões de como o comportamento complexo pode ser baseado em
processos evolutivos neuroanatômicos e neurobiológicos muito diferentes.

Revisitando, experimentalmente, a anatomia do cérebro odontoceto
(especificamente o do golfinho-nariz-de-garrafa), \citeonline{main} são capazes
de substanciar parte da base teórica biológica que tenta explicar as
capacidades cognitivas dos hominídeos, através de similaridades traçadas por
meio de análise da configuração citoarquitetônica do córtex cerebral de membros
da espécie.

Apesar das capacidades notáveis dos cetáceos, é relativamente recentemente o
interesse em buscar fontes outras do que de nossos antepassados evolutivos para
pesquisar a respeito de nossa inteligência. Afinal, o legado das ideias de
Darwin sugere que é lógico que deva-se endereçar tais questões à nossa própria
linhagem evolutiva. Porém, o quociente de encefalização somado à sua história
evolutiva distinta demonstram que os cetáceos se fazem de excepcional interesse
a esse estudo.

Desde então, algumas pesquisas analisam a forma desses cérebros
\cite{gaskin1982ecology, aronson1988conservative}, tentando desvendar quais
partes do cérebro desses indivíduos seriam homólogas ao nosso próprio cérebro,
além de medidas mais significativas do que seria inteligência, num contexto
mais geral, para além da definição restritiva de que é "a capacidade de
resolução de problemas" \cite{pearce2013animal}.

Notavelmente, esses estudos sobre a anatomia dos cérebros desses animais
falham em descrever a maior parte do objeto de estudo, dadas as diferenças
apreciáveis em sua forma, que surgem devido aos milhões de anos que se
desenvolveram de forma distinta aos hominídeos.

\citeonline{main}, então, analisam a citoarquitetura como também a arquitetura
química de cérebros dos golfinhos-nariz-de-garrafa a fim de desvendar essas
semelhanças, e também buscando entender o aparente paradoxo entre sua falta de
diferenciação e comportamento complexo.

O golfinho-nariz-de-garrafa mostra-se um excelente espécime para essa análise,
visto que, além de seu tamanho mais tratável em relação à uma baleira, possuem
a capacidade de se reconhecerem no espelho \cite{reiss2001mirror} e até de
usarem ferramentas \cite{krutzen2005cultural}, o que indica que são pelo menos
tão capazes quanto chimpanzés \cite{herman2002exploring}.

Seus resultados mostram que a aparente falta de diferenciação é produto de uma
plausível pedomorfose presente na infraordem dos cetáceos, em que se manifestam
traços de cérebros menos diferenciados, nomeadamente, àqueles de
características juvenis ancestrais.

O cérebro desses golfinhos demonstra suas próprias características de
especialização, próximas de artiodáctilos terrestres de grande porte, e que
indicam uma abordagem evolutiva distinta no processamento cortical dos
cetáceos.

No entanto, os cérebros desses animais demonstram notória homologia aos de outros mamíferos:

\begin{displayquote}
``Our findings reveal nonetheless that there are potentially as many neocortical
    regions that can be identified by cytoarchitectural criteria in cetaceans
    as in other mammals such as primates and carnivores.'' - \citeonline{main}
\end{displayquote}

Havendo demonstrado a complexidade por trás de diversos comportamentos
complexos desses animais, resta agora outra questão também levantada por
\cite{main}:

\begin{displayquote}
``What remains a com pelling question for future study, however, is how such
    dissimilar neocortical cytoarchitectural motifs, such as that found in
    cetaceans and primates, result in convergent cognitive and behavioral
    characteristics, and why.'' - \citeonline{main}
\end{displayquote}

Isto é, como suas características dissimilares levam à convergência de
comportamentos complexos.

\postextual

\bibliography{essay}

\printindex

\end{document}
